\documentclass[sigplan,anonymous,review]{acmart}

%include polycode.fmt

% \bibliographystyle{plainurl}% the mandatory bibstyle

\title{Modular Compilation using Effects}

\author{Jaro S. Reinders} % {Delft University of Technology, Netherlands}{j.s.reinders@tudelft.nl}{https://orcid.org/0000-0002-6837-3757}{} % (Optional) author-specific funding acknowledgements}

\begin{document}

\begin{abstract}

\end{abstract}

\maketitle

\section{Introduction (1 page)} \label{sec:intro}

Modern compilers for mature programming languages commonly consist of hundreds of thousands to tens of millions lines of code.
At the same time, programming languages are still very actively developed.
In the last 12 years many new programming languages have appeared, such as Kotlin, Rust, and Swift.
Rewriting all those lines of code for each new language is very costly.
It is not a coincidence that all three languages are built on top of mature back ends such as the JVM and LLVM.

However, these mature back ends all employ a ``one size fits all'' approach.
That means that new languages still need to write their own language specific optimization and adapter code for these mature back ends.


Concretely, our contributions are:
\begin{itemize}
  \item We build a compiler for a small language in Section~\ref{sec:simple-lang}.
  \item We show how to extend our compiler with conditionals and loops in Section~\ref{sec:cond-loops}.
  \item We evaluate our approach in Section~\ref{sec:evaluation}
\end{itemize}

\section{Effects and Handlers (1 page)} \label{sec:effects-handlers}

In this section we show what effects and handlers are and how they can be used in the context of compilers.

\section{Compiling Simple Language to X86 (3 pages)} \label{sec:simple-lang}

In this section we introduce a simple language and show how it can be compiled to X86 with variables.

\section{Register Allocation (2 pages)} \label{sec:analysis}

In this section we show how to run an analysis on our language.

\section{Adding Conditionals and Loops (2 pages)} \label{sec:cond-loops}

In this section we show that we can extend our language with new concepts without modifying code we have already written.

\section{Evaluation (2 pages)} \label{sec:evaluation}

In this section we evaluate the correctness of our compilers.

\section{Conclusion (1 page)} \label{sec:conclusion}



\end{document}